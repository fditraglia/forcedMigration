% Options for packages loaded elsewhere
\PassOptionsToPackage{unicode}{hyperref}
\PassOptionsToPackage{hyphens}{url}
%
\documentclass[
]{article}
\usepackage{amsmath,amssymb}
\usepackage{lmodern}
\usepackage{ifxetex,ifluatex}
\ifnum 0\ifxetex 1\fi\ifluatex 1\fi=0 % if pdftex
  \usepackage[T1]{fontenc}
  \usepackage[utf8]{inputenc}
  \usepackage{textcomp} % provide euro and other symbols
\else % if luatex or xetex
  \usepackage{unicode-math}
  \defaultfontfeatures{Scale=MatchLowercase}
  \defaultfontfeatures[\rmfamily]{Ligatures=TeX,Scale=1}
\fi
% Use upquote if available, for straight quotes in verbatim environments
\IfFileExists{upquote.sty}{\usepackage{upquote}}{}
\IfFileExists{microtype.sty}{% use microtype if available
  \usepackage[]{microtype}
  \UseMicrotypeSet[protrusion]{basicmath} % disable protrusion for tt fonts
}{}
\makeatletter
\@ifundefined{KOMAClassName}{% if non-KOMA class
  \IfFileExists{parskip.sty}{%
    \usepackage{parskip}
  }{% else
    \setlength{\parindent}{0pt}
    \setlength{\parskip}{6pt plus 2pt minus 1pt}}
}{% if KOMA class
  \KOMAoptions{parskip=half}}
\makeatother
\usepackage{xcolor}
\IfFileExists{xurl.sty}{\usepackage{xurl}}{} % add URL line breaks if available
\IfFileExists{bookmark.sty}{\usepackage{bookmark}}{\usepackage{hyperref}}
\hypersetup{
  pdftitle={plottingFunctions.R},
  pdfauthor={G1PXS05},
  hidelinks,
  pdfcreator={LaTeX via pandoc}}
\urlstyle{same} % disable monospaced font for URLs
\usepackage[margin=1in]{geometry}
\usepackage{color}
\usepackage{fancyvrb}
\newcommand{\VerbBar}{|}
\newcommand{\VERB}{\Verb[commandchars=\\\{\}]}
\DefineVerbatimEnvironment{Highlighting}{Verbatim}{commandchars=\\\{\}}
% Add ',fontsize=\small' for more characters per line
\usepackage{framed}
\definecolor{shadecolor}{RGB}{248,248,248}
\newenvironment{Shaded}{\begin{snugshade}}{\end{snugshade}}
\newcommand{\AlertTok}[1]{\textcolor[rgb]{0.94,0.16,0.16}{#1}}
\newcommand{\AnnotationTok}[1]{\textcolor[rgb]{0.56,0.35,0.01}{\textbf{\textit{#1}}}}
\newcommand{\AttributeTok}[1]{\textcolor[rgb]{0.77,0.63,0.00}{#1}}
\newcommand{\BaseNTok}[1]{\textcolor[rgb]{0.00,0.00,0.81}{#1}}
\newcommand{\BuiltInTok}[1]{#1}
\newcommand{\CharTok}[1]{\textcolor[rgb]{0.31,0.60,0.02}{#1}}
\newcommand{\CommentTok}[1]{\textcolor[rgb]{0.56,0.35,0.01}{\textit{#1}}}
\newcommand{\CommentVarTok}[1]{\textcolor[rgb]{0.56,0.35,0.01}{\textbf{\textit{#1}}}}
\newcommand{\ConstantTok}[1]{\textcolor[rgb]{0.00,0.00,0.00}{#1}}
\newcommand{\ControlFlowTok}[1]{\textcolor[rgb]{0.13,0.29,0.53}{\textbf{#1}}}
\newcommand{\DataTypeTok}[1]{\textcolor[rgb]{0.13,0.29,0.53}{#1}}
\newcommand{\DecValTok}[1]{\textcolor[rgb]{0.00,0.00,0.81}{#1}}
\newcommand{\DocumentationTok}[1]{\textcolor[rgb]{0.56,0.35,0.01}{\textbf{\textit{#1}}}}
\newcommand{\ErrorTok}[1]{\textcolor[rgb]{0.64,0.00,0.00}{\textbf{#1}}}
\newcommand{\ExtensionTok}[1]{#1}
\newcommand{\FloatTok}[1]{\textcolor[rgb]{0.00,0.00,0.81}{#1}}
\newcommand{\FunctionTok}[1]{\textcolor[rgb]{0.00,0.00,0.00}{#1}}
\newcommand{\ImportTok}[1]{#1}
\newcommand{\InformationTok}[1]{\textcolor[rgb]{0.56,0.35,0.01}{\textbf{\textit{#1}}}}
\newcommand{\KeywordTok}[1]{\textcolor[rgb]{0.13,0.29,0.53}{\textbf{#1}}}
\newcommand{\NormalTok}[1]{#1}
\newcommand{\OperatorTok}[1]{\textcolor[rgb]{0.81,0.36,0.00}{\textbf{#1}}}
\newcommand{\OtherTok}[1]{\textcolor[rgb]{0.56,0.35,0.01}{#1}}
\newcommand{\PreprocessorTok}[1]{\textcolor[rgb]{0.56,0.35,0.01}{\textit{#1}}}
\newcommand{\RegionMarkerTok}[1]{#1}
\newcommand{\SpecialCharTok}[1]{\textcolor[rgb]{0.00,0.00,0.00}{#1}}
\newcommand{\SpecialStringTok}[1]{\textcolor[rgb]{0.31,0.60,0.02}{#1}}
\newcommand{\StringTok}[1]{\textcolor[rgb]{0.31,0.60,0.02}{#1}}
\newcommand{\VariableTok}[1]{\textcolor[rgb]{0.00,0.00,0.00}{#1}}
\newcommand{\VerbatimStringTok}[1]{\textcolor[rgb]{0.31,0.60,0.02}{#1}}
\newcommand{\WarningTok}[1]{\textcolor[rgb]{0.56,0.35,0.01}{\textbf{\textit{#1}}}}
\usepackage{graphicx}
\makeatletter
\def\maxwidth{\ifdim\Gin@nat@width>\linewidth\linewidth\else\Gin@nat@width\fi}
\def\maxheight{\ifdim\Gin@nat@height>\textheight\textheight\else\Gin@nat@height\fi}
\makeatother
% Scale images if necessary, so that they will not overflow the page
% margins by default, and it is still possible to overwrite the defaults
% using explicit options in \includegraphics[width, height, ...]{}
\setkeys{Gin}{width=\maxwidth,height=\maxheight,keepaspectratio}
% Set default figure placement to htbp
\makeatletter
\def\fps@figure{htbp}
\makeatother
\setlength{\emergencystretch}{3em} % prevent overfull lines
\providecommand{\tightlist}{%
  \setlength{\itemsep}{0pt}\setlength{\parskip}{0pt}}
\setcounter{secnumdepth}{-\maxdimen} % remove section numbering
\ifluatex
  \usepackage{selnolig}  % disable illegal ligatures
\fi

\title{plottingFunctions.R}
\author{G1PXS05}
\date{2021-12-09}

\begin{document}
\maketitle

Helper function to run Dijkstra given parameters and create
corresponding distance dataframe. @param metric Distance metric being
used. Metric 0 corresponds to hops in the graph, metric 1 to crow-flies
distance, metric 2 to PCA without roads, and metric 3 to PCA with roads.
@param a Parameter for first principal component. @param b Parameter for
second principal component. @param epicenter\_1 First epicenter for
violence origin (municipality code), allowing us to create models of
outward spread from different origin pairs. @param epicenter\_2 Second
epicenter for violence origin (municipality code).

@return Dataframe containing distance, ring and municipality code.
Dataframe containing Municipality code, Year, Share, Ring, and Standard
Deviation.

\itemize{
   \item \code{total_dist} Distance between given pair of adjacent municipalities under metric. 
 }

\begin{Shaded}
\begin{Highlighting}[]
\NormalTok{generate\_distances }\OtherTok{\textless{}{-}} \ControlFlowTok{function}\NormalTok{(metric,a,b,epicenter\_1,epicenter\_2)\{}

\CommentTok{\# Figure out which PCA version to use (roads or no roads)}

\NormalTok{cross\_section\_merged }\OtherTok{\textless{}{-}} \FunctionTok{read.csv}\NormalTok{(}\StringTok{"data/full\_geographic\_covariates"}\NormalTok{)}
  
\NormalTok{pca }\OtherTok{\textless{}{-}} \FunctionTok{read.csv}\NormalTok{(}\StringTok{"data/pca"}\NormalTok{)}
\ControlFlowTok{if}\NormalTok{(metric }\SpecialCharTok{==} \DecValTok{3}\NormalTok{)\{}
\NormalTok{  pca }\OtherTok{\textless{}{-}} \FunctionTok{read.csv}\NormalTok{(}\StringTok{"data/pca\_roads"}\NormalTok{)}
\NormalTok{\}}

\CommentTok{\# Shift PCA\textquotesingle{}s to positive range. }
\NormalTok{pca}\SpecialCharTok{$}\NormalTok{PC1 }\OtherTok{\textless{}{-}}\NormalTok{ pca}\SpecialCharTok{$}\NormalTok{PC1}\SpecialCharTok{{-}}\FunctionTok{min}\NormalTok{(pca}\SpecialCharTok{$}\NormalTok{PC1)}
\NormalTok{pca}\SpecialCharTok{$}\NormalTok{PC2 }\OtherTok{\textless{}{-}}\NormalTok{ pca}\SpecialCharTok{$}\NormalTok{PC2}\SpecialCharTok{{-}}\FunctionTok{min}\NormalTok{(pca}\SpecialCharTok{$}\NormalTok{PC2)}
\NormalTok{pca}\SpecialCharTok{$}\NormalTok{PC1 }\OtherTok{\textless{}{-}}\NormalTok{ pca}\SpecialCharTok{$}\NormalTok{PC1 }\SpecialCharTok{/}\NormalTok{ (}\FunctionTok{max}\NormalTok{(pca}\SpecialCharTok{$}\NormalTok{PC1))}
\NormalTok{pca}\SpecialCharTok{$}\NormalTok{PC2 }\OtherTok{\textless{}{-}}\NormalTok{ pca}\SpecialCharTok{$}\NormalTok{PC2 }\SpecialCharTok{/}\NormalTok{ (}\FunctionTok{max}\NormalTok{(pca}\SpecialCharTok{$}\NormalTok{PC2))}

\CommentTok{\# Generate distances for each adjacent pair of municipalities. }
\ControlFlowTok{for}\NormalTok{(i }\ControlFlowTok{in} \DecValTok{1}\SpecialCharTok{:}\DecValTok{1120}\NormalTok{)\{}
  \ControlFlowTok{for}\NormalTok{(j }\ControlFlowTok{in} \DecValTok{1}\SpecialCharTok{:}\DecValTok{1120}\NormalTok{)\{}
    
    \CommentTok{\# Initially, set total\_dist (distance between municipalities i and j) to }
    \CommentTok{\# an effectively infinite number. }
\NormalTok{    total\_dist}\OtherTok{\textless{}{-}}\DecValTok{1000000}
    
    \CommentTok{\# Check that the municipalities are adjacent. }
    \ControlFlowTok{if}\NormalTok{(}\FunctionTok{are.connected}\NormalTok{(munigraph,i,j))\{}
      
      \CommentTok{\# If we have no geographic covariates for at least one municipality in the pair }
      \CommentTok{\# (and therefore did not calculate a PCA distance for this pair), set }
      \CommentTok{\# total\_dist to 1. (This is the "pure graph hops" case {-} we may want to revisit this later as an edge case). }
\NormalTok{      total\_dist }\OtherTok{\textless{}{-}} \DecValTok{1}
      
      \CommentTok{\# If using the crow{-}flies distance metric (or hiking, for cases where we }
      \CommentTok{\# lack elevation covariate), then set total\_dist to d1. }
      \ControlFlowTok{if}\NormalTok{(metric }\SpecialCharTok{==} \DecValTok{1} \SpecialCharTok{|}\NormalTok{ metric }\SpecialCharTok{==} \DecValTok{4} \SpecialCharTok{\&}\NormalTok{ i }\SpecialCharTok{\textless{}} \FunctionTok{nrow}\NormalTok{(cross\_section\_merged) }\SpecialCharTok{\&}\NormalTok{ j }\SpecialCharTok{\textless{}} \FunctionTok{nrow}\NormalTok{(cross\_section\_merged))\{}
        
        \CommentTok{\# Calculate crow{-}flies{-}distance. }
\NormalTok{        d1 }\OtherTok{\textless{}{-}} \FunctionTok{calcdist}\NormalTok{(cross\_section\_merged}\SpecialCharTok{$}\NormalTok{latnum[i],cross\_section\_merged}\SpecialCharTok{$}\NormalTok{lonnum[i]}
\NormalTok{                       ,cross\_section\_merged}\SpecialCharTok{$}\NormalTok{latnum[j],cross\_section\_merged}\SpecialCharTok{$}\NormalTok{lonnum[j])}
        
\NormalTok{        total\_dist }\OtherTok{\textless{}{-}}\NormalTok{ d1}
\NormalTok{      \}}
      
      \CommentTok{\# For distinct municipalities for which we have all covariates, calculate }
      \CommentTok{\# the distance. (For municipalities lacking geographic covariates, we }
      \CommentTok{\# default to crow distance). }
      
      \ControlFlowTok{if}\NormalTok{(metric}\SpecialCharTok{\textgreater{}}\DecValTok{1} \SpecialCharTok{\&}\NormalTok{ i}\SpecialCharTok{!=}\NormalTok{ j }\SpecialCharTok{\&}\NormalTok{ i}\SpecialCharTok{\textless{}}\FunctionTok{nrow}\NormalTok{(cross\_section\_merged) }\SpecialCharTok{\&}\NormalTok{ j}\SpecialCharTok{\textless{}}\FunctionTok{nrow}\NormalTok{(cross\_section\_merged))\{}
        \CommentTok{\# Retrieve principal components 1 and 2 for municipalities i and j from PCA dataframe.  }
\NormalTok{        pcode\_i }\OtherTok{\textless{}{-}}\NormalTok{ cross\_section\_merged}\SpecialCharTok{$}\NormalTok{ADM2\_PCODE[i]}
\NormalTok{        pcode\_j }\OtherTok{\textless{}{-}}\NormalTok{ cross\_section\_merged}\SpecialCharTok{$}\NormalTok{ADM2\_PCODE[j]}
\NormalTok{        row }\OtherTok{\textless{}{-}}\NormalTok{ pca[pca}\SpecialCharTok{$}\NormalTok{muni\_i }\SpecialCharTok{==}\NormalTok{ pcode\_i }\SpecialCharTok{\&}\NormalTok{ pca}\SpecialCharTok{$}\NormalTok{muni\_j }\SpecialCharTok{==}\NormalTok{ pcode\_j,]}\SpecialCharTok{$}\NormalTok{index}
\NormalTok{        PCA\_1 }\OtherTok{\textless{}{-}}\NormalTok{ pca}\SpecialCharTok{$}\NormalTok{PC1[row]}
\NormalTok{        PCA\_2 }\OtherTok{\textless{}{-}}\NormalTok{ pca}\SpecialCharTok{$}\NormalTok{PC2[row]}
        
        
        \CommentTok{\# Calculate total distance using our distance formula and save for summary statistics. }
\NormalTok{        total\_dist }\OtherTok{\textless{}{-}} \FunctionTok{exp}\NormalTok{((a}\SpecialCharTok{*}\NormalTok{PCA\_1}\SpecialCharTok{+}\NormalTok{b}\SpecialCharTok{*}\NormalTok{PCA\_2))}
\NormalTok{        summary\_db[row] }\OtherTok{\textless{}{-}}\NormalTok{ total\_dist}
        
        \CommentTok{\# If we are using the hiking metric (and both municipalities are among }
        \CommentTok{\# the 1,076 for which we have geographic covariates), calculate and }
        \CommentTok{\# update distance. Where we don\textquotesingle{}t have covariates, the default is d1.  }
        \ControlFlowTok{if}\NormalTok{(metric}\SpecialCharTok{==}\DecValTok{4} \SpecialCharTok{\&}\NormalTok{ i}\SpecialCharTok{\textless{}}\FunctionTok{nrow}\NormalTok{(cross\_section\_merged) }\SpecialCharTok{\&}\NormalTok{ j}\SpecialCharTok{\textless{}}\FunctionTok{nrow}\NormalTok{(cross\_section\_merged) )\{}
\NormalTok{          elev\_difference }\OtherTok{\textless{}{-}} \FunctionTok{max}\NormalTok{((cross\_section\_merged}\SpecialCharTok{$}\NormalTok{alt\_mean[i]}\SpecialCharTok{{-}}
\NormalTok{                                    cross\_section\_merged}\SpecialCharTok{$}\NormalTok{alt\_mean[j]),}\DecValTok{0}\NormalTok{)}
\NormalTok{          total\_dist }\OtherTok{\textless{}{-}}\NormalTok{ d1 }\SpecialCharTok{+} \FloatTok{0.6} \SpecialCharTok{*}\NormalTok{ elev\_difference}
\NormalTok{          summary\_db[row] }\OtherTok{\textless{}{-}}\NormalTok{ total\_dist}
\NormalTok{        \}}
\NormalTok{      \}}
      
      \CommentTok{\# Update and set edge weight. }
\NormalTok{      edge }\OtherTok{\textless{}{-}} \FunctionTok{get.edge.ids}\NormalTok{(munigraph,}\FunctionTok{c}\NormalTok{(i,j))}
\NormalTok{      munigraph }\OtherTok{\textless{}{-}} \FunctionTok{set\_edge\_attr}\NormalTok{(munigraph,}\StringTok{"weight"}\NormalTok{,edge,total\_dist)}
\NormalTok{    \}}
\NormalTok{  \}}
\NormalTok{\}}

\NormalTok{vertex\_ids }\OtherTok{=} \FunctionTok{vector}\NormalTok{(}\AttributeTok{length =} \DecValTok{1120}\NormalTok{)}
\ControlFlowTok{for}\NormalTok{(i }\ControlFlowTok{in} \DecValTok{1}\SpecialCharTok{:}\DecValTok{1120}\NormalTok{)\{}
\NormalTok{  vertex\_ids[i]}\OtherTok{\textless{}{-}}\NormalTok{ AttributeTableFinal}\SpecialCharTok{$}\NormalTok{ADM2\_PCODE[i]}
\NormalTok{\}}

\CommentTok{\# Calculate Dijkstra distances }
\NormalTok{delta\_1 }\OtherTok{=} \FunctionTok{distances}\NormalTok{(munigraph,}\AttributeTok{v =} \FunctionTok{which}\NormalTok{(vertex\_ids }\SpecialCharTok{==}\NormalTok{ epicenter\_1),}\AttributeTok{to =} \FunctionTok{V}\NormalTok{(munigraph),}\AttributeTok{algorithm =} \StringTok{"dijkstra"}\NormalTok{)}
\NormalTok{delta\_2 }\OtherTok{=} \FunctionTok{distances}\NormalTok{(munigraph,}\AttributeTok{v =} \FunctionTok{which}\NormalTok{(vertex\_ids }\SpecialCharTok{==}\NormalTok{ epicenter\_2),}\AttributeTok{to =} \FunctionTok{V}\NormalTok{(munigraph),}\AttributeTok{algorithm =} \StringTok{"dijkstra"}\NormalTok{)}
\NormalTok{deltas }\OtherTok{\textless{}{-}} \FunctionTok{data.frame}\NormalTok{(}\FunctionTok{matrix}\NormalTok{(}\FunctionTok{c}\NormalTok{(delta\_1,delta\_2),}\AttributeTok{ncol =} \DecValTok{2}\NormalTok{))}
\FunctionTok{colnames}\NormalTok{(deltas) }\OtherTok{\textless{}{-}} \FunctionTok{c}\NormalTok{(}\StringTok{"delta\_1"}\NormalTok{,}\StringTok{"delta\_2"}\NormalTok{);}
\NormalTok{deltas }\OtherTok{\textless{}{-}} \FunctionTok{mutate}\NormalTok{(deltas, }\AttributeTok{delta\_min =} \FunctionTok{as.numeric}\NormalTok{(purrr}\SpecialCharTok{::}\FunctionTok{map2}\NormalTok{(delta\_1,delta\_2,min)))}


\CommentTok{\# Merge our distances (from specified epicenter) into dataframe containing map polygons. }
\NormalTok{distances }\OtherTok{\textless{}{-}} \FunctionTok{as.numeric}\NormalTok{(deltas[,}\DecValTok{3}\NormalTok{])}


\CommentTok{\# Dataset for merging delta and ring\_num into other datasets by ADM2\_PCODE. }
\NormalTok{merge\_deltas }\OtherTok{\textless{}{-}} \FunctionTok{data.frame}\NormalTok{(}\AttributeTok{delta =}\NormalTok{ distances,}\AttributeTok{ADM2\_PCODE =}\NormalTok{ vertex\_ids)}

\CommentTok{\# Calculate ring as 10 * decile in distance distribution plus 1.  }
\NormalTok{merge\_deltas }\OtherTok{\textless{}{-}} \FunctionTok{mutate}\NormalTok{(merge\_deltas, }\AttributeTok{ring\_num =} \FunctionTok{as.integer}\NormalTok{(}\DecValTok{10}\SpecialCharTok{*}\FunctionTok{ecdf}\NormalTok{(merge\_deltas}\SpecialCharTok{$}\NormalTok{delta)(delta))}\SpecialCharTok{+}\DecValTok{1}\NormalTok{)}


\CommentTok{\# For the single largest delta in the dataset, decile is 10 and so ring is 11. }
\CommentTok{\# Set ring\_num to 10 for this case. }
\NormalTok{merge\_deltas }\OtherTok{\textless{}{-}} \FunctionTok{mutate}\NormalTok{(merge\_deltas, }\AttributeTok{ring\_num =} \FunctionTok{ifelse}\NormalTok{(ring\_num }\SpecialCharTok{==} \DecValTok{11}\NormalTok{,}\DecValTok{10}\NormalTok{,ring\_num))}
\FunctionTok{return}\NormalTok{(merge\_deltas)}
\NormalTok{\}}
\end{Highlighting}
\end{Shaded}

Helper function to create distances given epicenter and metric
parameters. @param merge\_deltas Output from generate\_distances;
contains mapping of municipality to distance and ring under given
metric, a, b, and epicenter.

@return Dataframe containing municipality, ring, share of total violence
for ring in given year, and standard deviation of violence across
municipalities in a given ring-year. \itemize{
\item \code{FinalWithDeltas} Dataframe containing geographic covariates and final distances. 
   \item \code{vdf} Dataframe containing year-ring violence. 
   \item \code{sdf} Dataframe containing standard errors of year-ring violence. 
   \item \code{sharedf} Dataframe containing final output. 
}

\begin{Shaded}
\begin{Highlighting}[]
\NormalTok{get\_df }\OtherTok{\textless{}{-}} \ControlFlowTok{function}\NormalTok{(merge\_deltas)\{}
\NormalTok{FinalWithDeltas }\OtherTok{\textless{}{-}} \FunctionTok{merge}\NormalTok{(cross\_section\_merged,merge\_deltas,}\AttributeTok{by =} \StringTok{"ADM2\_PCODE"}\NormalTok{)}
\NormalTok{violence\_data }\OtherTok{\textless{}{-}}\NormalTok{ violence\_data[violence\_data}\SpecialCharTok{$}\NormalTok{year }\SpecialCharTok{\textless{}} \DecValTok{2009}\NormalTok{,]}
\NormalTok{violence\_set }\OtherTok{\textless{}{-}} \FunctionTok{merge}\NormalTok{(violence\_data,FinalWithDeltas,}\AttributeTok{by=}\StringTok{"ADM2\_PCODE"}\NormalTok{)}

\CommentTok{\#print(summary(FinalWithDeltas$delta))}

\CommentTok{\# Take out municipalities with 0 violence. }

\NormalTok{violence\_subset }\OtherTok{\textless{}{-}} \FunctionTok{filter}\NormalTok{(violence\_set,year }\SpecialCharTok{==} \DecValTok{2008}\NormalTok{)}
\NormalTok{has\_violence\_list }\OtherTok{\textless{}{-}} \FunctionTok{filter}\NormalTok{(violence\_subset,cum\_victims\_UR }\SpecialCharTok{\textgreater{}}\DecValTok{0}\NormalTok{)}\SpecialCharTok{$}\NormalTok{ADM2\_PCODE}
\NormalTok{violence\_set }\OtherTok{\textless{}{-}} \FunctionTok{filter}\NormalTok{(violence\_set, ADM2\_PCODE }\SpecialCharTok{\%in\%}\NormalTok{ has\_violence\_list)}


\CommentTok{\# For each ring and year, stores total number of deaths in that ring in that year. }
\NormalTok{victimsUR\_matrix }\OtherTok{\textless{}{-}} \FunctionTok{matrix}\NormalTok{(}\DecValTok{0}\NormalTok{,}\AttributeTok{nrow =} \DecValTok{10}\NormalTok{,}\AttributeTok{ncol =} \DecValTok{13}\NormalTok{)}

\CommentTok{\# For each ring and year, stores standard deviation of the set \{deaths in municipality in year : municipality in ring\} . }
\NormalTok{stderr\_matrix }\OtherTok{\textless{}{-}} \FunctionTok{matrix}\NormalTok{(}\DecValTok{0}\NormalTok{,}\AttributeTok{nrow =} \DecValTok{10}\NormalTok{,}\AttributeTok{ncol =} \DecValTok{13}\NormalTok{)}

\CommentTok{\# Stores the two above quantities in their respective matrices. }
\ControlFlowTok{for}\NormalTok{(ring }\ControlFlowTok{in} \DecValTok{1}\SpecialCharTok{:}\DecValTok{10}\NormalTok{)\{}
  \ControlFlowTok{for}\NormalTok{(year }\ControlFlowTok{in} \DecValTok{1996}\SpecialCharTok{:}\DecValTok{2008}\NormalTok{)\{}
\NormalTok{    victimsUR\_matrix[ring,year}\DecValTok{{-}1995}\NormalTok{] }\OtherTok{\textless{}{-}} \FunctionTok{sum}\NormalTok{(violence\_set[violence\_set}\SpecialCharTok{$}\NormalTok{year }\SpecialCharTok{==}\NormalTok{ year }\SpecialCharTok{\&}\NormalTok{ violence\_set}\SpecialCharTok{$}\NormalTok{ring }\SpecialCharTok{==}\NormalTok{ ring,]}\SpecialCharTok{$}\NormalTok{victims\_\_UR)}
\NormalTok{    stderr\_matrix[ring,year}\DecValTok{{-}1995}\NormalTok{] }\OtherTok{\textless{}{-}} \FunctionTok{std.error}\NormalTok{(violence\_set[violence\_set}\SpecialCharTok{$}\NormalTok{year }\SpecialCharTok{==}\NormalTok{ year }\SpecialCharTok{\&}\NormalTok{ violence\_set}\SpecialCharTok{$}\NormalTok{ring }\SpecialCharTok{==}\NormalTok{ ring,]}\SpecialCharTok{$}\NormalTok{victims\_\_UR)}
\NormalTok{  \}}
\NormalTok{\}}

\CommentTok{\# Convert matrices to dataframes. }
\NormalTok{vdf }\OtherTok{\textless{}{-}} \FunctionTok{as.data.frame}\NormalTok{(}\FunctionTok{t}\NormalTok{(victimsUR\_matrix))}
\NormalTok{sdf }\OtherTok{\textless{}{-}} \FunctionTok{as.data.frame}\NormalTok{(stderr\_matrix)}

\CommentTok{\# Generate share of total ring deaths that occur in a given year by dividing each ring{-}year death figure by total ring deaths. }
\NormalTok{sharedf }\OtherTok{\textless{}{-}} \FunctionTok{mutate}\NormalTok{(vdf,}\AttributeTok{V1 =}\NormalTok{ V1}\SpecialCharTok{/}\FunctionTok{sum}\NormalTok{(V1),}\AttributeTok{V2 =}\NormalTok{ V2}\SpecialCharTok{/}\FunctionTok{sum}\NormalTok{(V2),}\AttributeTok{V3 =}\NormalTok{ V3}\SpecialCharTok{/}\FunctionTok{sum}\NormalTok{(V3),}\AttributeTok{V4 =}\NormalTok{ V4}\SpecialCharTok{/}\FunctionTok{sum}\NormalTok{(V4), }\AttributeTok{V5 =}\NormalTok{ V5}\SpecialCharTok{/}\FunctionTok{sum}\NormalTok{(V5), }\AttributeTok{V6 =}\NormalTok{ V6}\SpecialCharTok{/}\FunctionTok{sum}\NormalTok{(V6), }\AttributeTok{V7 =}\NormalTok{ V7}\SpecialCharTok{/}\FunctionTok{sum}\NormalTok{(V7), }\AttributeTok{V8 =}\NormalTok{ V8}\SpecialCharTok{/}\FunctionTok{sum}\NormalTok{(V8), }\AttributeTok{V9 =}\NormalTok{ V9}\SpecialCharTok{/}\FunctionTok{sum}\NormalTok{(V9), }\AttributeTok{V10 =}\NormalTok{ V10}\SpecialCharTok{/}\FunctionTok{sum}\NormalTok{(V10))}
\NormalTok{sharedf }\OtherTok{\textless{}{-}} \FunctionTok{as.data.frame}\NormalTok{(}\FunctionTok{t}\NormalTok{(sharedf))}

\NormalTok{cnames }\OtherTok{=} \FunctionTok{paste}\NormalTok{(}\DecValTok{1996}\SpecialCharTok{:}\DecValTok{2008}\NormalTok{)}
\FunctionTok{colnames}\NormalTok{(sharedf) }\OtherTok{\textless{}{-}}\NormalTok{ cnames}
\FunctionTok{colnames}\NormalTok{(sdf) }\OtherTok{\textless{}{-}}\NormalTok{ cnames}
\FunctionTok{rownames}\NormalTok{(sharedf) }\OtherTok{\textless{}{-}} \DecValTok{1}\SpecialCharTok{:}\DecValTok{10}
\FunctionTok{rownames}\NormalTok{(sdf) }\OtherTok{\textless{}{-}} \DecValTok{1}\SpecialCharTok{:}\DecValTok{10}

\CommentTok{\# Since the rows represent rings, explicitly create a ring variable corresponding to row number. }
\NormalTok{sharedf }\OtherTok{\textless{}{-}} \FunctionTok{cbind}\NormalTok{(}\StringTok{"ring"} \OtherTok{=} \FunctionTok{as.numeric}\NormalTok{(}\FunctionTok{rownames}\NormalTok{(sharedf)),sharedf)}

\CommentTok{\# Make dataframes long to allow for plotting. }
\NormalTok{sharedf }\OtherTok{\textless{}{-}} \FunctionTok{pivot\_longer}\NormalTok{(sharedf,cnames,}\AttributeTok{names\_to =} \StringTok{"year"}\NormalTok{,}\AttributeTok{values\_to =} \StringTok{"share"}\NormalTok{)}
\NormalTok{sdf }\OtherTok{\textless{}{-}} \FunctionTok{cbind}\NormalTok{(}\StringTok{"ring"} \OtherTok{=} \FunctionTok{as.numeric}\NormalTok{(}\FunctionTok{rownames}\NormalTok{(sdf)),sdf)}
\NormalTok{sdf }\OtherTok{\textless{}{-}} \FunctionTok{pivot\_longer}\NormalTok{(sdf,cnames,}\AttributeTok{names\_to =} \StringTok{"year"}\NormalTok{,}\AttributeTok{values\_to =} \StringTok{"sd"}\NormalTok{)}
\NormalTok{sharedf }\OtherTok{\textless{}{-}} \FunctionTok{merge}\NormalTok{(sharedf,sdf,}\AttributeTok{by =} \FunctionTok{c}\NormalTok{(}\StringTok{"ring"}\NormalTok{,}\StringTok{"year"}\NormalTok{))}

\FunctionTok{return}\NormalTok{(sharedf)}

\NormalTok{\}  }
\end{Highlighting}
\end{Shaded}

Heatmap plot to visualize distribution of violence in rings over time.

@param metric Distance metric being used. Metric 0 corresponds to hops
in the graph, metric 1 to crow-flies distance, metric 2 to PCA without
roads, and metric 3 to PCA with roads. @param a Parameter for first
principal component. @param b Parameter for second principal component.
@param epicenter\_1 First epicenter for violence origin (municipality
code), allowing us to create models of outward spread from different
origin pairs. @param epicenter\_2 Second epicenter for violence origin
(municipality code).

@return NULL (plots corresponding graph).

\begin{Shaded}
\begin{Highlighting}[]
\NormalTok{heatmap }\OtherTok{\textless{}{-}} \ControlFlowTok{function}\NormalTok{(metric,a,b,epicenter\_1,epicenter\_2)\{}
\NormalTok{merged\_deltas }\OtherTok{\textless{}{-}} \FunctionTok{generate\_distances}\NormalTok{(metric,a,b,epicenter\_1,epicenter\_2)}
\NormalTok{sharedf }\OtherTok{\textless{}{-}} \FunctionTok{get\_df}\NormalTok{(merged\_deltas)}
\FunctionTok{ggplot}\NormalTok{(}\AttributeTok{data=}\NormalTok{sharedf,}\AttributeTok{mapping=}\FunctionTok{aes}\NormalTok{(}\AttributeTok{x=}\NormalTok{year,}\AttributeTok{y=}\NormalTok{ring,}\AttributeTok{fill=}\NormalTok{share))}\SpecialCharTok{+}
  \FunctionTok{geom\_tile}\NormalTok{()}\SpecialCharTok{+}\FunctionTok{theme\_minimal}\NormalTok{()}\SpecialCharTok{+}\FunctionTok{scale\_fill\_gradient}\NormalTok{(}\AttributeTok{name=}\StringTok{"Violence Share"}\NormalTok{,}\AttributeTok{low=}\StringTok{"darkblue"}\NormalTok{,}\AttributeTok{high=}\StringTok{"red"}\NormalTok{)}\SpecialCharTok{+}
  \FunctionTok{scale\_x\_discrete}\NormalTok{(}\AttributeTok{breaks =} \FunctionTok{seq}\NormalTok{(}\DecValTok{1}\NormalTok{,}\DecValTok{10}\NormalTok{,}\DecValTok{3}\NormalTok{),}\AttributeTok{label =} \FunctionTok{paste}\NormalTok{(}\FunctionTok{seq}\NormalTok{(}\DecValTok{1}\NormalTok{,}\DecValTok{10}\NormalTok{,}\DecValTok{3}\NormalTok{)))}
\NormalTok{\}}
\end{Highlighting}
\end{Shaded}

Star Wars plot (without bars) to visualize violence over time within
rings.

@param metric Distance metric being used. Metric 0 corresponds to hops
in the graph, metric 1 to crow-flies distance, metric 2 to PCA without
roads, and metric 3 to PCA with roads. @param a Parameter for first
principal component. @param b Parameter for second principal component.
@param epicenter\_1 First epicenter for violence origin (municipality
code), allowing us to create models of outward spread from different
origin pairs. @param epicenter\_2 Second epicenter for violence origin
(municipality code).

@return NULL (plots corresponding graph).

\begin{Shaded}
\begin{Highlighting}[]
\NormalTok{starwars }\OtherTok{\textless{}{-}}  \ControlFlowTok{function}\NormalTok{(metric,a,b,epicenter\_1,epicenter\_2)\{}
\NormalTok{merged\_deltas }\OtherTok{\textless{}{-}} \FunctionTok{generate\_distances}\NormalTok{(metric,a,b,epicenter\_1,epicenter\_2)}
\NormalTok{sharedf }\OtherTok{\textless{}{-}} \FunctionTok{get\_df}\NormalTok{(merged\_deltas)}
\FunctionTok{ggplot}\NormalTok{(sharedf,}\FunctionTok{aes}\NormalTok{(}\AttributeTok{x =}\NormalTok{ year,}\AttributeTok{y =}\NormalTok{ share,}\AttributeTok{group =} \FunctionTok{factor}\NormalTok{(ring)))}\SpecialCharTok{+}\FunctionTok{geom\_point}\NormalTok{(}\FunctionTok{aes}\NormalTok{(}\AttributeTok{colour =} \FunctionTok{factor}\NormalTok{(ring)))}\SpecialCharTok{+}\FunctionTok{facet\_wrap}\NormalTok{(}\SpecialCharTok{\textasciitilde{}}\NormalTok{ ring)}
\NormalTok{\}}
\end{Highlighting}
\end{Shaded}

Star Wars plot (with bars) to visualize violence over time within rings.

@param metric Distance metric being used. Metric 0 corresponds to hops
in the graph, metric 1 to crow-flies distance, metric 2 to PCA without
roads, and metric 3 to PCA with roads. @param a Parameter for first
principal component. @param b Parameter for second principal component.
@param epicenter\_1 First epicenter for violence origin (municipality
code), allowing us to create models of outward spread from different
origin pairs. @param epicenter\_2 Second epicenter for violence origin
(municipality code).

@return NULL (plots corresponding graph).

\begin{Shaded}
\begin{Highlighting}[]
\NormalTok{starwars\_bars }\OtherTok{\textless{}{-}} \ControlFlowTok{function}\NormalTok{(metric,a,b,epicenter\_1,epicenter\_2)\{}
\NormalTok{merged\_deltas }\OtherTok{\textless{}{-}} \FunctionTok{generate\_distances}\NormalTok{(metric,a,b,epicenter\_1,epicenter\_2)}
\NormalTok{sharedf }\OtherTok{\textless{}{-}} \FunctionTok{get\_df}\NormalTok{(merged\_deltas)}
\FunctionTok{ggplot}\NormalTok{(sharedf,}\FunctionTok{aes}\NormalTok{(}\AttributeTok{x =}\NormalTok{ year,}\AttributeTok{y =}\NormalTok{ share,}\AttributeTok{group =} \FunctionTok{factor}\NormalTok{(ring)))}\SpecialCharTok{+}\FunctionTok{geom\_point}\NormalTok{(}\FunctionTok{aes}\NormalTok{(}\AttributeTok{colour =} \FunctionTok{factor}\NormalTok{(ring)))}\SpecialCharTok{+}\FunctionTok{geom\_errorbar}\NormalTok{(}\FunctionTok{aes}\NormalTok{(}\AttributeTok{ymin =}\NormalTok{ share}\SpecialCharTok{{-}}\NormalTok{sd, }\AttributeTok{ymax =}\NormalTok{ share}\SpecialCharTok{+}\NormalTok{sd,}\AttributeTok{colour =} \FunctionTok{factor}\NormalTok{(ring)))}\SpecialCharTok{+}\FunctionTok{facet\_wrap}\NormalTok{(}\SpecialCharTok{\textasciitilde{}}\NormalTok{ ring)}
\NormalTok{\}}
\end{Highlighting}
\end{Shaded}

Spatial map plot to visualize distances.

@param metric Distance metric being used. Metric 0 corresponds to hops
in the graph, metric 1 to crow-flies distance, metric 2 to PCA without
roads, and metric 3 to PCA with roads. @param a Parameter for first
principal component. @param b Parameter for second principal component.
@param epicenter\_1 First epicenter for violence origin (municipality
code), allowing us to create models of outward spread from different
origin pairs. @param epicenter\_2 Second epicenter for violence origin
(municipality code).

@return NULL (plots corresponding graph).

\begin{Shaded}
\begin{Highlighting}[]
\NormalTok{spatial\_map }\OtherTok{\textless{}{-}} \ControlFlowTok{function}\NormalTok{(metric,a,b,epicenter\_1,epicenter\_2)\{}
\NormalTok{merged\_deltas }\OtherTok{\textless{}{-}} \FunctionTok{generate\_distances}\NormalTok{(metric,a,b,epicenter\_1,epicenter\_2)}
\NormalTok{muni\_pol }\OtherTok{\textless{}{-}} \FunctionTok{st\_read}\NormalTok{(}\StringTok{"Data/col muni polygons/col\_admbnda\_adm2\_mgn\_20200416.shp"}\NormalTok{)}
\NormalTok{muni\_pol }\OtherTok{\textless{}{-}} \FunctionTok{select}\NormalTok{(muni\_pol,}\StringTok{"ADM2\_PCODE"}\NormalTok{)}
\NormalTok{merged\_map }\OtherTok{\textless{}{-}} \FunctionTok{merge}\NormalTok{(merged\_deltas,muni\_pol)}
\FunctionTok{ggplot}\NormalTok{() }\SpecialCharTok{+}
  \FunctionTok{geom\_sf}\NormalTok{(}\AttributeTok{data =} \FunctionTok{st\_as\_sf}\NormalTok{(merged\_map), }\FunctionTok{aes}\NormalTok{(}\AttributeTok{fill=}\NormalTok{ring\_num),}\AttributeTok{color =} \StringTok{\textquotesingle{}grey34\textquotesingle{}}\NormalTok{,}\AttributeTok{lwd=}\NormalTok{.}\DecValTok{05}\NormalTok{) }\SpecialCharTok{+}
  \FunctionTok{scale\_fill\_stepsn}\NormalTok{(}\AttributeTok{colors =} \FunctionTok{c}\NormalTok{(}\StringTok{"red"}\NormalTok{,}\StringTok{"gold"}\NormalTok{,}\StringTok{"darkgreen"}\NormalTok{,}\StringTok{"blue"}\NormalTok{,}\StringTok{"violet"}\NormalTok{),}\AttributeTok{values =} \ConstantTok{NULL}\NormalTok{,}\AttributeTok{space =} \StringTok{"Lab"}\NormalTok{,}\AttributeTok{na.value =} \StringTok{"grey50"}\NormalTok{,}\AttributeTok{guide =} \StringTok{"coloursteps"}\NormalTok{,}\AttributeTok{aesthetics =} \StringTok{"fill"}\NormalTok{,}\AttributeTok{breaks =} \FunctionTok{c}\NormalTok{(}\DecValTok{1}\SpecialCharTok{:}\DecValTok{10}\NormalTok{),}\AttributeTok{limits=} \FunctionTok{c}\NormalTok{(}\DecValTok{1}\NormalTok{, }\DecValTok{10}\NormalTok{))}
\NormalTok{\}}
\end{Highlighting}
\end{Shaded}


\end{document}
